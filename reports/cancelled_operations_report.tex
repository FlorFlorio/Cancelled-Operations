% Options for packages loaded elsewhere
\PassOptionsToPackage{unicode}{hyperref}
\PassOptionsToPackage{hyphens}{url}
%
\documentclass[
]{article}
\usepackage{lmodern}
\usepackage{amssymb,amsmath}
\usepackage{ifxetex,ifluatex}
\ifnum 0\ifxetex 1\fi\ifluatex 1\fi=0 % if pdftex
  \usepackage[T1]{fontenc}
  \usepackage[utf8]{inputenc}
  \usepackage{textcomp} % provide euro and other symbols
\else % if luatex or xetex
  \usepackage{unicode-math}
  \defaultfontfeatures{Scale=MatchLowercase}
  \defaultfontfeatures[\rmfamily]{Ligatures=TeX,Scale=1}
  \setmainfont[]{Arial}
\fi
% Use upquote if available, for straight quotes in verbatim environments
\IfFileExists{upquote.sty}{\usepackage{upquote}}{}
\IfFileExists{microtype.sty}{% use microtype if available
  \usepackage[]{microtype}
  \UseMicrotypeSet[protrusion]{basicmath} % disable protrusion for tt fonts
}{}
\makeatletter
\@ifundefined{KOMAClassName}{% if non-KOMA class
  \IfFileExists{parskip.sty}{%
    \usepackage{parskip}
  }{% else
    \setlength{\parindent}{0pt}
    \setlength{\parskip}{6pt plus 2pt minus 1pt}}
}{% if KOMA class
  \KOMAoptions{parskip=half}}
\makeatother
\usepackage{xcolor}
\IfFileExists{xurl.sty}{\usepackage{xurl}}{} % add URL line breaks if available
\IfFileExists{bookmark.sty}{\usepackage{bookmark}}{\usepackage{hyperref}}
\hypersetup{
  pdftitle={ Análisis de series temporales de operaciones canceladas},
  pdfauthor={Lic. Malena Álvarez Brito, Ing. Florencia Florio},
  hidelinks,
  pdfcreator={LaTeX via pandoc}}
\urlstyle{same} % disable monospaced font for URLs
\usepackage[margin=1in]{geometry}
\usepackage{longtable,booktabs}
% Correct order of tables after \paragraph or \subparagraph
\usepackage{etoolbox}
\makeatletter
\patchcmd\longtable{\par}{\if@noskipsec\mbox{}\fi\par}{}{}
\makeatother
% Allow footnotes in longtable head/foot
\IfFileExists{footnotehyper.sty}{\usepackage{footnotehyper}}{\usepackage{footnote}}
\makesavenoteenv{longtable}
\usepackage{graphicx,grffile}
\makeatletter
\def\maxwidth{\ifdim\Gin@nat@width>\linewidth\linewidth\else\Gin@nat@width\fi}
\def\maxheight{\ifdim\Gin@nat@height>\textheight\textheight\else\Gin@nat@height\fi}
\makeatother
% Scale images if necessary, so that they will not overflow the page
% margins by default, and it is still possible to overwrite the defaults
% using explicit options in \includegraphics[width, height, ...]{}
\setkeys{Gin}{width=\maxwidth,height=\maxheight,keepaspectratio}
% Set default figure placement to htbp
\makeatletter
\def\fps@figure{htbp}
\makeatother
\setlength{\emergencystretch}{3em} % prevent overfull lines
\providecommand{\tightlist}{%
  \setlength{\itemsep}{0pt}\setlength{\parskip}{0pt}}
\setcounter{secnumdepth}{5}
\usepackage{fancyhdr}
\pagestyle{fancy}
\fancyhead[LE,RO]{M. Álvarez Brito, F. Florio}

\title{\includegraphics[width=1in,height=\textheight]{logo_dm_uba.png}\\
Análisis de series temporales de operaciones canceladas}
\author{Lic. Malena Álvarez Brito, Ing. Florencia Florio}
\date{octubre 2020}

\begin{document}
\maketitle

\begin{quote}
\textbf{Resumen}: Este es el resumen. Este es el resumen. Este es el
resumen. Este es el resumen. Este es el resumen. Este es el resumen.
Este es el resumen. Este es el resumen. Este es el resumen. Este es el
resumen. Este es el resumen. Este es el resumen. Este es el resumen.
(aproximadamente 250 palabras).
\end{quote}

\begin{quote}
Palabras clave: series de tiempo, sistema de salud, forecasting
\end{quote}

\hypertarget{introducciuxf3n}{%
\section{Introducción}\label{introducciuxf3n}}

Introducción. Explicar de dónde viene el dataset. Qué significa cada
valor. Periodicidad de la serie (frecuencia de muestreo). Qué se
entiende por operación ``electiva''. Qué tipo de instituciones son?
públicas y privadas? O sólo públicas?

\hypertarget{hipuxf3tesis-y-objetivos}{%
\section{Hipótesis y objetivos}\label{hipuxf3tesis-y-objetivos}}

Hipótesis y objetivos.

Objetivo: analizar la serie de tiempo de operaciones canceladas para
descubrir patrones en su comportamiento, ver si existe correlación con
ocupación de las camas de internación, hacer forecasting de porcentaje
de cancelación. Todo esto podría contribuir a una mejor asignación de
recursos del sistema de salud y toma de decisión de gestión sanitaria.

\hypertarget{material-y-muxe9todos}{%
\section{Material y métodos}\label{material-y-muxe9todos}}

Material y métodos.

R, RStudio. Librerías: tidyverse, zoo, xts, forecast, seasonal.

\hypertarget{resultados}{%
\section{Resultados}\label{resultados}}

Gráfico temporal de todas las series en el dataset de operaciones
canceladas.

\hypertarget{respecto-al-porcentaje-de-cancelaciones}{%
\subsubsection{Respecto al porcentaje de
cancelaciones}\label{respecto-al-porcentaje-de-cancelaciones}}

Fuerte estacionalidad (sube, baja, sube, baja). Algo pasó en 2001 que
aumentó el porcentaje de cancelaciones (casi el doble que en otros
trimestres). Además, entre el 2000 y el 2010 parecería haber una
tendencia a disminuir. Luego, en 2010 parece haber un cambio de
tendencia y empieza a aumentar ligeramente.

\hypertarget{respecto-al-total-de-opraciones}{%
\subsubsection{Respecto al total de
opraciones}\label{respecto-al-total-de-opraciones}}

Clara tendencia a aumentar.

\includegraphics{cancelled_operations_report_files/figure-latex/series-1.pdf}

Viendo el filtro moving average también se evidencia el aumento en 2001.

\begin{verbatim}
## Warning: Removed 4 row(s) containing missing values (geom_path).
\end{verbatim}

\includegraphics{cancelled_operations_report_files/figure-latex/moving-average-1.pdf}

\hypertarget{anuxe1lisis-a-partir-de-los-gruxe1ficos-de-estacionalidad}{%
\subsubsection{Análisis a partir de los gráficos de
estacionalidad}\label{anuxe1lisis-a-partir-de-los-gruxe1ficos-de-estacionalidad}}

En el siguiente gráfico se ve que el pico fue en realidad en el último
trimestre de 2000 pero que todo el 2000 fue alto. También el 2001 fue
alto, aunque luego fue bajando.

Yo veo, además, que el primer y segundo trimestres suelen ser más bajos
que el tercero y el cuarto. El 2010 fue distinto porque el Q4 fue
bastante menor al Q3.

\includegraphics{cancelled_operations_report_files/figure-latex/seasonal-plot-year-1.pdf}

En el siguiente gráfico se ve que los primeros dos trimestres son, en
promedio, más bajos que el tercero y el cuarto.

\begin{center}\includegraphics{cancelled_operations_report_files/figure-latex/seasonal-plot-quarter-1} \end{center}

Autocorrelación

\begin{center}\includegraphics{cancelled_operations_report_files/figure-latex/acf-1} \end{center}

Gráfico de la fft de porcentaje de operaciones canceladas. Se distinguen
2 picos bien marcados (a parte del valor del índice 1\ldots este sería
el valor de continua???)

\begin{center}\includegraphics{cancelled_operations_report_files/figure-latex/fft-1} \end{center}

\begin{longtable}[]{@{}rr@{}}
\toprule
valor & posicion\tabularnewline
\midrule
\endhead
1677577.0 & 1\tabularnewline
136952.8 & 3\tabularnewline
119513.9 & 27\tabularnewline
\bottomrule
\end{longtable}

Descomposición de la señal. El cuarto gráfico es como el de moving
average. Después podemos ver con cuál nos quedamos.

\includegraphics{cancelled_operations_report_files/figure-latex/classical-decomposition-1.pdf}

\hypertarget{forecasting}{%
\subsection{Forecasting}\label{forecasting}}

Forecast simple (preliminar).

\includegraphics{cancelled_operations_report_files/figure-latex/forecast-1.pdf}

\hypertarget{auto-arima-forecast}{%
\subsubsection{Auto ARIMA forecast}\label{auto-arima-forecast}}

Se aplica un ARIMA automático con estacionalidad.

\begin{verbatim}
## Series: tt 
## ARIMA(1,0,0)(0,1,1)[4] 
## 
## Coefficients:
##          ar1     sma1
##       0.8637  -0.8829
## s.e.  0.0546   0.0618
## 
## sigma^2 estimated as 9.364e-07:  log likelihood=544.63
## AIC=-1083.25   AICc=-1083   BIC=-1075.47
\end{verbatim}

\includegraphics{cancelled_operations_report_files/figure-latex/auto-arima-1.pdf}

\hypertarget{correlaciuxf3n-con-ocupaciuxf3n-de-camas}{%
\subsection{Correlación con ocupación de
camas}\label{correlaciuxf3n-con-ocupaciuxf3n-de-camas}}

\hypertarget{correlaciuxf3n-con-ocupaciuxf3n-de-camas-por-el-duxeda.}{%
\paragraph{Correlación con ocupación de camas por el
día.}\label{correlaciuxf3n-con-ocupaciuxf3n-de-camas-por-el-duxeda.}}

Mirando el gráfico de dispersión, no parece haber una relación tan clara
entre ambas.

\begin{center}\includegraphics{cancelled_operations_report_files/figure-latex/corr-day-beds-1} \end{center}

\begin{verbatim}
## `geom_smooth()` using formula 'y ~ x'
\end{verbatim}

\begin{center}\includegraphics{cancelled_operations_report_files/figure-latex/corr-day-beds-2} \end{center}

\begin{center}\includegraphics{cancelled_operations_report_files/figure-latex/corr-day-beds-3} \end{center}

\hypertarget{correlaciuxf3n-con-camas-duxedanoche.}{%
\paragraph{Correlación con camas
día+noche.}\label{correlaciuxf3n-con-camas-duxedanoche.}}

En el gráfico de dispersión se ve una relación entre ambas. En decir,
cuando hay mucha ocupación de camas nocturnas, hay mayor porcentaje de
cancelación de operaciones. Esta relación (aparentemente lineal) no
implica causalidad.

\begin{center}\includegraphics{cancelled_operations_report_files/figure-latex/corr-overnight-beds-1} \end{center}

\begin{verbatim}
## `geom_smooth()` using formula 'y ~ x'
\end{verbatim}

\begin{center}\includegraphics{cancelled_operations_report_files/figure-latex/corr-overnight-beds-2} \end{center}

\begin{center}\includegraphics{cancelled_operations_report_files/figure-latex/corr-overnight-beds-3} \end{center}

\hypertarget{discusiuxf3n-y-conclusiones}{%
\section{Discusión y conclusiones}\label{discusiuxf3n-y-conclusiones}}

\hypertarget{bibliografuxeda}{%
\section{Bibliografía}\label{bibliografuxeda}}

\end{document}
