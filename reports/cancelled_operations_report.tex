% Options for packages loaded elsewhere
\PassOptionsToPackage{unicode}{hyperref}
\PassOptionsToPackage{hyphens}{url}
%
\documentclass[
]{article}
\usepackage{lmodern}
\usepackage{amssymb,amsmath}
\usepackage{ifxetex,ifluatex}
\ifnum 0\ifxetex 1\fi\ifluatex 1\fi=0 % if pdftex
  \usepackage[T1]{fontenc}
  \usepackage[utf8]{inputenc}
  \usepackage{textcomp} % provide euro and other symbols
\else % if luatex or xetex
  \usepackage{unicode-math}
  \defaultfontfeatures{Scale=MatchLowercase}
  \defaultfontfeatures[\rmfamily]{Ligatures=TeX,Scale=1}
  \setmainfont[]{Arial}
\fi
% Use upquote if available, for straight quotes in verbatim environments
\IfFileExists{upquote.sty}{\usepackage{upquote}}{}
\IfFileExists{microtype.sty}{% use microtype if available
  \usepackage[]{microtype}
  \UseMicrotypeSet[protrusion]{basicmath} % disable protrusion for tt fonts
}{}
\makeatletter
\@ifundefined{KOMAClassName}{% if non-KOMA class
  \IfFileExists{parskip.sty}{%
    \usepackage{parskip}
  }{% else
    \setlength{\parindent}{0pt}
    \setlength{\parskip}{6pt plus 2pt minus 1pt}}
}{% if KOMA class
  \KOMAoptions{parskip=half}}
\makeatother
\usepackage{xcolor}
\IfFileExists{xurl.sty}{\usepackage{xurl}}{} % add URL line breaks if available
\IfFileExists{bookmark.sty}{\usepackage{bookmark}}{\usepackage{hyperref}}
\hypersetup{
  pdftitle={ Análisis de series temporales de operaciones canceladas en Inglaterra},
  pdfauthor={Lic. Malena Álvarez Brito, Ing. Florencia Florio},
  hidelinks,
  pdfcreator={LaTeX via pandoc}}
\urlstyle{same} % disable monospaced font for URLs
\usepackage[margin=1in]{geometry}
\usepackage{longtable,booktabs}
% Correct order of tables after \paragraph or \subparagraph
\usepackage{etoolbox}
\makeatletter
\patchcmd\longtable{\par}{\if@noskipsec\mbox{}\fi\par}{}{}
\makeatother
% Allow footnotes in longtable head/foot
\IfFileExists{footnotehyper.sty}{\usepackage{footnotehyper}}{\usepackage{footnote}}
\makesavenoteenv{longtable}
\usepackage{graphicx,grffile}
\makeatletter
\def\maxwidth{\ifdim\Gin@nat@width>\linewidth\linewidth\else\Gin@nat@width\fi}
\def\maxheight{\ifdim\Gin@nat@height>\textheight\textheight\else\Gin@nat@height\fi}
\makeatother
% Scale images if necessary, so that they will not overflow the page
% margins by default, and it is still possible to overwrite the defaults
% using explicit options in \includegraphics[width, height, ...]{}
\setkeys{Gin}{width=\maxwidth,height=\maxheight,keepaspectratio}
% Set default figure placement to htbp
\makeatletter
\def\fps@figure{htbp}
\makeatother
\setlength{\emergencystretch}{3em} % prevent overfull lines
\providecommand{\tightlist}{%
  \setlength{\itemsep}{0pt}\setlength{\parskip}{0pt}}
\setcounter{secnumdepth}{5}
\usepackage{fancyhdr}
\pagestyle{fancy}
\fancyhead[LE,RO]{M. Álvarez Brito, F. Florio}

\title{\includegraphics[width=1in,height=\textheight]{logo_dm_uba.png}\\
Análisis de series temporales de operaciones canceladas en Inglaterra}
\author{Lic. Malena Álvarez Brito, Ing. Florencia Florio}
\date{octubre 2020}

\begin{document}
\maketitle

\begin{quote}
\textbf{Resumen}: En la presente investigacioón aplicamos distintas
tecnicas con el fin de evaluar patrones de comportamiento en las
cancelaciones de operaciones en Gran Bretaña. La información de la
ocupación de camas nos parecia un dato muy relevante para agregar al
análisis asi como tambien para entender el comportamiento de los datos.
En su desarrollo hemos aplicado tecnicas de forecating con el fin de
predecir el porcentaje de cancelación de operaciones lo que ayudaría
principalmente a la correcta planifiación de los recursos en el sistema
de salud britanico.
\end{quote}

\begin{quote}
Palabras clave: series de tiempo, sistema de salud, forecasting
\end{quote}

\hypertarget{introducciuxf3n}{%
\section{Introducción}\label{introducciuxf3n}}

Los quirófanos representan un costo considerable para los centros de
salud dado que requieren una gran cantidad de recursos disponibles para
garantizar su funcionamiento. Por este motivo, la subutilización de los
quirófanos por parte de los centros de salud, va en detrimento de las
finanzas del centro. En la presente investigación hemos trabajado con
datos publicados por el servicio de salud de Inglaterra acerca de la
cancelación a último minuto de operaciones electivas en hospitales
públicos.

\hypertarget{hipuxf3tesis-y-objetivos}{%
\section{Hipótesis y objetivos}\label{hipuxf3tesis-y-objetivos}}

El objetivo del presente trabajo es analizar la serie de tiempo de
operaciones electivas canceladas para descubrir patrones en su
comportamiento y realizar, de ser posible, un forecasting de valores
futuros en el porcentaje de cancelaciones. También se busca analizar qué
tanto contribuye el porcentaje de ocupación de camas trimestrales en los
hospitales públicos al porcentaje de cancelación de operaciones
electivas. Se cree que la variable ocupación de camas contribuye al
aumento de la cancelaciones.

El resultado de este tipo de análisis podría contribuir a una mejor
asignación de recursos del sistema de salud y toma de decisión de
gestión sanitaria.

\hypertarget{material-y-muxe9todos}{%
\section{Material y métodos}\label{material-y-muxe9todos}}

Los datos utilizados corresponden a aquellos publicados por el servicio
de salud de Inglaterra acerca de las operaciones electivas en centros de
salud públicos \footnote{\url{https://www.england.nhs.uk/statistics/statistical-work-areas/cancelled-elective-operations/}}.
El dataset consta de datos trimestrales sobre:

\begin{itemize}
\tightlist
\item
  Cantidad de operaciones electivas programadas (no incluye
  procedimientos ambulatorios menores).
\item
  Cantidad de cancelaciones de operaciones a último minuto por parte del
  hospital debido a motivos no clínicos.
\item
  Cantidad de pacientes no tratados en un lapso 28 días posteriores a la
  cancelación (este punto no fue incorporado al análisis).
\end{itemize}

Ejemplos de motivos de cancelaciones (son no clínicos):

\begin{itemize}
\tightlist
\item
  Cirujano no disponible
\item
  Anestesista no disponible
\item
  Otro personal no disponible
\item
  Cama de sala no disponible
\item
  Cama de cuidados intensivos no disponible
\item
  Caso de emergencia que necesita quirófano
\item
  Falla en el equipamiento médico
\item
  Error administrativo
\end{itemize}

Dado que el dataset no aporta estadísticas sobre la prevalencia de los
motivos de cancelación, se utiliza otro data set provisto por el
servicio de salud de Inglaterra referido al porcentaje de ocupación de
las camas de internación en sus hospitales públicos \footnote{\url{https://www.england.nhs.uk/statistics/statistical-work-areas/bed-availability-and-occupancy/}}
para analizar qué tanto aporta esta variable a la cancelación. El data
set de ocupación se encuentra diferenciado por camas para internación
diurna o nocturna.

El análisis de series temporales tiene utilidad en el estudio de la
relación entre variables que cambian con el tiempo y que pueden llegar a
influenciarse entre sí. Los datos se pueden comportar de diferentes
maneras a través del tiempo: puede que se presente una tendencia,
estacionalidad o simplemente no presenten una forma definida.

Se utilizaron los siguientes conceptos:

\begin{itemize}
\tightlist
\item
  Descomposición de series temporales
\item
  Análisis en el dominio de la frecuencia
\item
  Autocorrelación de una serie temporal
\item
  Forecasting
\item
  Correlación entre series de tiempo
\end{itemize}

La investigación fue realizada con R versión 4.0.2 en el IDE RStudio.
Las principales librerías utilizadas fueron: tidyverse, zoo, xts,
forecast, seasonal. Para el presente informe se utilizó RMarkdown.

\hypertarget{resultados}{%
\section{Resultados}\label{resultados}}

Se analiza la cancelación de operaciones como porcentaje del total de
operaciones para quitarle la tendencia ascendente que tiene esta última.
Se observa que dicho porcentaje tiene un pico en 2001 (casi el doble que
en otros trimestres) y depresión entre 2010 y 2012. En la descomposición
multiplicativa, la curva suavizada resultante muestra dichas
oscilaciones no estacionales. También, se evidencia la fuerte
estacionalidad de los datos.

\includegraphics{cancelled_operations_report_files/figure-latex/series-1.pdf}

\includegraphics{cancelled_operations_report_files/figure-latex/classical-decomposition-1.pdf}

Comparando los ciclos anuales, se observa que el pico ocurrió
específicamente en el último trimestre de 2000 pero que todo el 2000 fue
atípicamente alto. También el 2001 fue alto, aunque luego fue bajando.
Tanto en este gráfico como en el de subestacionalidad se ve que los
primeros dos trimestres son, en promedio, más bajos que el tercero y el
cuarto.

\includegraphics{cancelled_operations_report_files/figure-latex/seasonal-plot-year-1.pdf}

\begin{center}\includegraphics{cancelled_operations_report_files/figure-latex/unnamed-chunk-1-1} \end{center}

Respecto a la autocorrelación, se ve que el pico más alto está en 4. Los
picos tienden a tener una separación de 4 (observaciones) trimestres y
van disminuyendo en su amplitud. Forman picos debido a la
estacionalidad. El hecho de que la correlación sea positiva en casi
todos los lags se debe a que la serie tiene una tendencia positiva.

\begin{center}\includegraphics{cancelled_operations_report_files/figure-latex/acf-1} \end{center}

Gráfico de la fft de porcentaje de operaciones canceladas. Se distinguen
2 picos bien marcados (a parte del valor del índice 1.

\begin{center}\includegraphics{cancelled_operations_report_files/figure-latex/fft-1} \end{center}

\begin{longtable}[]{@{}rr@{}}
\toprule
Valor & Frecuencia\tabularnewline
\midrule
\endhead
1677577 & 1\tabularnewline
136953 & 3\tabularnewline
119514 & 27\tabularnewline
\bottomrule
\end{longtable}

\hypertarget{forecasting}{%
\subsection{Forecasting}\label{forecasting}}

Para el forecasting se aplica un ARIMA automático con estacionalidad.

\begin{verbatim}
## Series: tt 
## ARIMA(1,0,0)(0,1,1)[4] 
## 
## Coefficients:
##          ar1     sma1
##       0.8637  -0.8829
## s.e.  0.0546   0.0618
## 
## sigma^2 estimated as 0.009364:  log likelihood=88.72
## AIC=-171.43   AICc=-171.18   BIC=-163.65
\end{verbatim}

\includegraphics{cancelled_operations_report_files/figure-latex/auto-arima-1.pdf}

Residuos del forecast:

\includegraphics{cancelled_operations_report_files/figure-latex/auto-arima-residuals-1.pdf}

\hypertarget{correlaciuxf3n-con-ocupaciuxf3n-de-camas}{%
\subsection{correlación con ocupación de
camas}\label{correlaciuxf3n-con-ocupaciuxf3n-de-camas}}

\hypertarget{ocupaciuxf3n-de-camas-diurnas}{%
\subsubsection{Ocupación de camas
(Diurnas)}\label{ocupaciuxf3n-de-camas-diurnas}}

\begin{center}\includegraphics{cancelled_operations_report_files/figure-latex/corr-day-beds-1} \end{center}

Correlación lineal con Ocupación de camas (Diurnas).

\begin{verbatim}
## `geom_smooth()` using formula 'y ~ x'
\end{verbatim}

\begin{center}\includegraphics{cancelled_operations_report_files/figure-latex/corr-day-beds2-1} \end{center}

Correlación temporal con Ocupación de camas (Diurnas):

Un aumento (o disminución) en el \% de ocupación de las camas diurnas
por sobre la media, tiene correlación con un aumento (o disminución) en
el \% de cancelación de operaciones 1 año (4 muestras) después.

\begin{center}\includegraphics{cancelled_operations_report_files/figure-latex/corr-day-beds3-1} \end{center}

\hypertarget{ocupaciuxf3n-de-camas-nocturnas}{%
\subsubsection{Ocupación de camas
(Nocturnas)}\label{ocupaciuxf3n-de-camas-nocturnas}}

La correlación temporal entre las señales no muestra un claro anticipo
de una a la otra.

\begin{center}\includegraphics{cancelled_operations_report_files/figure-latex/corr-overnight-beds-1} \end{center}

Correlación lineal con Ocupación de camas (Nocturnas).

\begin{verbatim}
## `geom_smooth()` using formula 'y ~ x'
\end{verbatim}

\begin{center}\includegraphics{cancelled_operations_report_files/figure-latex/corr-overnight-beds2-1} \end{center}

Correlación temporal con Ocupación de camas (Nocturnas).

\begin{center}\includegraphics{cancelled_operations_report_files/figure-latex/corr-overnight-beds3-1} \end{center}

\hypertarget{discusiuxf3n-y-conclusiones}{%
\section{Discusión y conclusiones}\label{discusiuxf3n-y-conclusiones}}

La principal característica encontrada en la serie temporal de
porcentaje de cancelación de operaciones electivas es su marcada
estacionalidad. Tiene un periodo anual, dentro del cual se evidencia un
mayor porcentaje de cancelación en la segunda mitad del año respecto a
la primera mitad. El pico anual se suele alcanzar en el último trimestre
del año. Este fenómeno podría estar explicado por un aumento en la
ocupación de las camas que se da en la época invernal.

También parecería existir un ciclo, no asociado al calendario, más
amplio. Además, existe una elevación atípica alrededor del año 2001 y
una depresión alrededor del 2011. Globalmente, la serie de cancelaciones
presenta una ligera tendencia a aumentar.

El modelo de forecasting resultó satisfactorio ya que los residuos se
aproximan a una distribución normal y prácticamente no muestran
autocorrelación, es decir que se parecen al ruido blanco (aleatorio).
Existió solamente un pico significativo en la autocorrelación de los
residuos. Además, estos se ven menos ``aleatorios'' alrededor de las
irregularidades de la serie en los años 2001 y 2011. Estos últimos
hallazgos en los residuos del modelo dan cuenta de que existe una
pequeña parte de la información de la serie que no logró ser capturada
por el modelo.

Respecto a la correlación con la ocupación de camas diurnas, se observa
que no existe una clara relación lineal pero sí temporal. Sin embargo,
al incorporar la información de las camas de noche si se observa una
fuerte relación lineal, pero no una clara relación temporal con las
cancelaciones de operaciones.

Se concluye que el conjunto de técnicas aplicadas fuer lo
suficientemente sensible como para detectar patrones interesantes en los
datos así como también verificar la hipótesis de la correlación entre la
ocupación de camas y el aumento de la cancelación de operaciones.

Como mejoras al proyecto, se sugiere realizar alguna transformación
sobre la serie previo a aplicar el modelo ARIMA para que este pueda
capturar por completo sus cualidades.

Sería interesante, además, incorporar variables internas de la
administración sanitaria, así como también de tipo contextuales del país
en estudio.

Futuramente, se intentará replicar este estudio en datos de la República
Argentina con el fin de contribuir a mejorar la planificación de los
recursos en la gestión del sistema de salud.

\hypertarget{bibliografuxeda}{%
\section{Bibliografía}\label{bibliografuxeda}}

Hyndman, R.J., \& Athanasopoulos, G. (2018) Forecasting: principles and
practice, 2nd edition, OTexts: Melbourne, Australia.
\href{OTexts.com/fpp2}{https://otexts.com/fpp2/}. Accedido el 22 de
octubre de 2020.

Theopahno Mitsa (2010) Temporal Data Mining, Chapman and Hall/CRC

\end{document}
